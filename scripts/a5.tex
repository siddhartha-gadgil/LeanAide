\documentclass{amsart}
\theoremstyle{plain}
\newtheorem{question}{Question}

\begin{document}

\begin{question}
    Find all $2 \times 2$ matrices with rational entries such that $A^5 = I$, where $I$ is the $2 \times 2$ identity matrix.
\end{question}

\begin{proof}
    A $2 \times 2$ matrix $A$ over the rationals is of the form

\[
A = \begin{pmatrix}
a & b \\
c & d 
\end{pmatrix},
\]

where $a, b, c, d \in \mathbb{Q}$. We want to find all such $A$ such that $A^5 = I$, where $I$ is the $2 \times 2$ identity matrix. 

We can start by noting that the minimal polynomial of such an $A$ must divide $x^5 - 1 = 0$. Over the rationals, this polynomial factors into irreducible factors as

\[
x^5 - 1 = (x - 1)(x^4 + x^3 + x^2 + x + 1).
\]

So the minimal polynomial of $A$ must either be $x - 1$ or $x^4 + x^3 + x^2 + x + 1 = 0$. However, the second polynomial has degree $4$ and the minimal polynomial of a $2 \times 2$ matrix can have degree at most $2$, so $A$ must satisfy $x - 1 = 0$. That is, $A = I$. 

To verify our claim, we note that if $A = I$, then indeed $A^5 = I^5 = I$. So the only $2 \times 2$ matrix over the rationals such that $A^5 = I$ is $A = I$.
\end{proof}

\begin{proof}The condition $A^5 = I$ implies that the minimal polynomial of $A$ divides $x^5 - 1 = 0$. The factorization of $x^5 - 1$ over the rational numbers is $(x-1)(x^4+x^3+x^2+x+1) = 0$. 

The minimal polynomial of a $2 \times 2$ matrix is at most of degree 2. Therefore, the minimal polynomial of $A$ can either be $x-1=0$, $x^2-1=0$, $x^2+x+1=0$, or $x^4+x^3+x^2+x+1=0$.

1. If the minimal polynomial is $x-1=0$, then $A=I$ is the only solution.

2. If the minimal polynomial is $x^2-1=0$, then the solutions correspond to matrices $A$ which are similar to the diagonal matrix with entries $1,-1$. The matrices similar to a diagonal matrix with entries $1,-1$ are of the form $\begin{pmatrix} a & b \ c & -a \end{pmatrix}$ such that $a^2 - bc = 1$.

3. If the minimal polynomial is $x^2+x+1=0$, then the solutions correspond to matrices $A$ which are similar to the diagonal matrix with entries $\omega, \omega^2$, where $\omega$ is a primitive cube root of unity. However, these matrices cannot have rational entries.

4. If the minimal polynomial is $x^4+x^3+x^2+x+1=0$, it cannot be the minimal polynomial of a $2 \times 2$ matrix since its degree is greater than 2.

In summary, the solutions are all $2 \times 2$ matrices of the form $\begin{pmatrix} a & b \ c & -a \end{pmatrix}$ such that $a^2 - bc = 1$ and the $2 \times 2$ identity matrix. This solution is complete.
\end{proof}

\begin{proof}To find all 2x2 matrices with rational entries such that $A^5 = I$, where $I$ is the 2x2 identity matrix, we need to consider the cyclic subgroup generated by A under matrix multiplication, which has order 5.

Firstly, observe that the minimal polynomial of such a matrix $A$ must divide $x^5 - 1 = 0$.
Decomposing this into irreducible polynomials over the rationals (using the fact that it is a cyclotomic polynomial), we get $x^5 - 1 = (x-1)(x^4 + x^3 + x^2 + x + 1)$.

The roots of $x-1$ are 1, which corresponds to the identity matrix. The roots of $x^4 + x^3 + x^2 + x + 1$ are the 4 complex 5th roots of unity which are not 1, namely $e^{2\pi i / 5}, e^{4\pi i / 5}, e^{6\pi i / 5}$, and $e^{8\pi i / 5}$.

Since the minimal polynomial of a 2x2 matrix must be of degree at most 2, the minimal polynomial of $A$ can be either $(x-1)$, $(x - e^{2\pi i / 5})(x - e^{8\pi i / 5})$, or $(x - e^{4\pi i / 5})(x - e^{6\pi i / 5})$. The case of $(x-1)$ corresponds to the identity matrix.

In the other two cases, the matrices corresponding will be similar to the matrix:

$A = \begin{bmatrix}
a & -b\
b & a
\end{bmatrix}$

for the two cases where (a, b) are elements of either $(cos(2\pi/5), sin(2\pi/5))$ or $(cos(4\pi/5), sin(4\pi/5))$. The other two roots are negatives of these two and will give the same matrices (up to similarity).

To prove this, note that any 2x2 matrix with a minimal polynomial of degree 2 is similar to a matrix in this form, and that a matrix in this form has the desired properties. Also, any 2x2 matrix satisfying $A^5 = I$ has a minimal polynomial dividing $x^5 - 1$, and so its minimal polynomial must be one of the three cases we have listed.

So finally, the matrices $A$ in question are the $2 \times 2$ matrices with rational entries which are similar to either the identity matrix, or one of the two matrices corresponding to $cos(2\pi/5)$, $sin(2\pi/5)$ or $cos(4\pi/5)$, $sin(4\pi/5)$.
\end{proof}


\begin{question}
    Prove that the polynomial $x^4 + x^3 + x^2 + x + 1$ is irreducible over the rationals.
\end{question}

\begin{proof}To show that the polynomial $x^4 + x^3 + x^2 + x + 1$ is irreducible over the rationals, we can use the following strategy. 

First, we make a substitution $y = x+1$, so that the polynomial becomes $y^4 - C$ where $C$ is a constant (specifically, $C = 5 + 4y + 6y^2 + 4y^3$). 

Now it's clear that if the polynomial were reducible, its factors would have to be of the form $(y^2 + ay + b)(y^2 + cy + d)$ for some rational numbers $a, b, c, d$. But then we would have $ac=0$, (since $y^4$ has no $y^3$ term), which implies that either $a=0$ or $c=0$.

However, if either $a=0$ or $c=0$, then the constant term of one of the factors would have to be a square root of $C$, but $C$ is not a perfect square, so this is impossible. Hence the polynomial is irreducible over the rationals. 

The underlying principle here is Eisenstein's Criterion, which states that a polynomial is irreducible if there exists a prime number $p$ such that $p$ divides all coefficients of the polynomial except the leading one, $p^2$ does not divide the constant term, and the leading coefficient is not divisible by $p$. In this case, the prime number is $p=1$, and the coefficients of the polynomial are all 1.\end{proof}
\begin{proof}There are many ways to show that a polynomial is irreducible over the rationals, but a particularly useful one here is a concept of Eisenstein's Criterion. However, Eisenstein's Criterion is not directly applicable for this polynomial. 

One way we can apply it is through what's called a change of variables. We need to find a suitable substitution that transforms the polynomial into a form where Eisenstein's Criterion applies. 

Let's let $y = x+1$. Then we have
$y^4 + y^3 + y^2 + y + 1 = (x+1)^4 + (x+1)^3 + (x+1)^2 + (x+1) + 1 = x^4 + 5x^3 + 10x^2 + 10x + 5.$ 

Now we can apply Eisenstein's Criterion with $p = 5$ to conclude that $x^4 + 5x^3 + 10x^2 + 10x + 5$ is irreducible over the rationals.

Because the roots of the original polynomial $x^4 + x^3 + x^2 + x + 1$ are precisely the roots of $x^4 + 5x^3 + 10x^2 + 10x + 5$ decreased by 1, we can conclude that the original polynomial is also irreducible over the rationals.

Eisenstein's Criterion states that if you have a polynomial 
\[a_nx^n + a_{n-1}x^{n-1} + ... + a_1x + a_0\]
and there exists a prime number $p$ such that:
1. $p$ divides $a_i$ for all $i < n$,
2. $p$ does not divide $a_n$,
3. $p^2$ does not divide $a_0$,

then the polynomial is irreducible over the rationals.\end{proof}
\begin{proof}To prove that the polynomial $x^4 + x^3 + x^2 + x + 1$ is irreducible over the rationals, one approach is by using a substitution and then applying Eisenstein's Criterion, a well-known theorem in number theory that provides a sufficient condition for a polynomial to be irreducible over the rational numbers.

Let the polynomial be denoted as $f(x) = x^4 + x^3 + x^2 + x + 1$. Now consider the substitution $x = y-1$, that is, replace $x$ with $(y-1)$ in the polynomial given. This gives us a new polynomial:

$g(y) = (y-1)^4 + (y-1)^3 + (y-1)^2 + (y-1) + 1 = y^4 - 4y^3 + 6y^2 - 4y + 1$.

Now, observe that all the coefficients of $g$ are integers, and every coefficient of a term whose degree is less than 4 is divisible by 4, but $4^2$ does not divide the constant term 1. Furthermore, the constant term 1 is a prime number. 

Therefore, by Eisenstein's Criterion (with prime number p=4), the polynomial $g(y)$ is irreducible over Q.

Finally, because $f(x)$ and $g(y)$ are related by a simple substitution $(y=x+1)$, it can be said that if $g(y)$ is irreducible over Q, then $f(x)$ is also irreducible over Q. Hence, the polynomial $x^4 + x^3 + x^2 + x + 1$ is irreducible over Q.\end{proof}

\begin{question}
    Find all finite subgroups $G$ such that $G$ has exactly three subgroups.
\end{question}
\begin{proof}A finite group $G$ has at least two subgroups: the trivial group $\{e\}$ and the group $G$ itself.

Now if the group $G$ has exactly three subgroups, then there must exist exactly one non-trivial proper subgroup, say $H$. However, since $H$ is a subgroup, it must also contain the trivial group and itself. This means $H$ must be a group such that it has no other proper subgroup, i.e., $H$ must be a simple group.

The only finite simple groups that do not have any proper subgroups other than the trivial group and itself are the cyclic groups of prime order. Thus, $H$ must be a cyclic group of prime order, say $p$. 

So, we have $H \cong \mathbb{Z}_p$ for some prime $p$. Then, the group $G$ must be such that it has a subgroup isomorphic to $\mathbb{Z}_p$ but no other proper subgroups. Hence, the group $G$ must be isomorphic to $\mathbb{Z}_p$ or $\mathbb{Z}_{p^2}$.

Now, we know that $\mathbb{Z}_p$ has exactly two subgroups, which contradicts our assumption that $G$ has exactly three subgroups. Therefore, $G$ cannot be isomorphic to $\mathbb{Z}_p$.

On the other hand, the group $\mathbb{Z}_{p^2}$ has exactly three subgroups: the trivial group, the whole group, and the subgroup of order $p$. Hence, $G$ must be isomorphic to $\mathbb{Z}_{p^2}$ for some prime $p$.

So, the solution to the problem is all groups isomorphic to $\mathbb{Z}_{p^2}$ for some prime number $p$.\end{proof}
\begin{proof}Let $G$ be a finite group with exactly three subgroups. Then we know that these three subgroups are $G$, the trivial subgroup $\{e\}$, and one non-trivial proper subgroup $H$. 

We know that the order of any subgroup must divide the order of the group by Lagrange's theorem. Let $|G|=n$ and $|H|=m$. Thus, $m|n$ and $1<m<n$.

Also, we know that for any group $G$, the center of the group $Z(G)$ is always a subgroup of $G$. If $Z(G)=G$, then $G$ must be abelian, and in this case, there must be more than 3 subgroups unless $G$ is cyclic. If $Z(G)=\{e\}$, then $G$ is non-abelian of order $n>1$, which also must have more than three subgroups. Thus, $Z(G)$ must be equal to $H$. 

In the case $Z(G)=H$, from the class equation we have the following:
$|G|=|Z(G)|+\sum |G:C_G(g)|$, where the sum is over one representative from each conjugacy class outside of $Z(G)$. 

Each $|G:C_G(g)|$ term divides $|G|=n$, so it is either $1$, $m$ or $n$. If $|G:C_G(g)|=1$ for some $g$, then $g$ is in $Z(G)=H$, which is a contradiction. If $|G:C_G(g)|=n$ for some $g$, then $|C_G(g)|=1$ which means $g$ is not in $Z(G)$ but commutes with all elements of $G$, hence $g$ must be in $Z(G)$ which is a contradiction. Thus, all $|G:C_G(g)|=m$ which means that $|G:C_G(g)|=|H|$.

Then we have $|G|=|Z(G)|+k|H|$ where $k$ is number of terms in the sum. Since $|G|=n$ and $|Z(G)|=|H|=m$, it follows that $n=m+km$. Therefore, $n=m(k+1)$ which implies $k+1|n$. But we know that $1<m<n$ and $k+1>m$, so we must have $k+1=n$ and $m=1$. 

There is only one group of order 1, the trivial group, so $H$ is the trivial group, contradicting our assumption that $H$ is non-trivial. Therefore, we have a contradiction. 

So, there is no finite group with exactly three subgroups.\end{proof}
\begin{proof}A finite group $G$ with exactly three subgroups must satisfy the following conditions:

1. $G$ itself is a group, so it is always one of the these three subgroups.

2. The trivial subgroup containing only the identity element is also always a subgroup, so it is the second of these three subgroups.

3. Finally, let's consider the third subgroup, which we'll call $H$. Since $H$ is neither $G$ nor the trivial subgroup, it must be a proper nonzero subgroup of $G$.

4. Since $H$ is the only proper nonzero subgroup of $G$, $H$ must be a maximal subgroup of $G$.

5. By Lagrange's Theorem, the order of $H$ must divide the order of $G$. 

Given these conditions, we can deduce that $G$ must be a cyclic group of prime square order. To see why, first note that if $|G|$ was not a prime square, then either $|G|$ would have additional divisors, meaning that $G$ would have additional subgroups, or $|G|$ would be prime, in which case $H$ would not exist since all proper subgroups of a prime order group are trivial.

Therefore, $|G|=p^2$ for some prime $p$. In this case, the subgroups of $G$ are the trivial group, $G$ itself, and a unique subgroup of order $p$. This is because the cyclic group of order $p^2$ has exactly one subgroup of order $p$ and no other proper subgroups.

Therefore, the finite groups with exactly three subgroups are exactly the cyclic groups of order $p^2$ for prime $p$.
\end{proof}
\begin{question}
    Find all finite subgroups $G$ such that there are exactly three conjugacy classes of elements in $G$.
\end{question}

\begin{proof}Any finite group $G$ with exactly three conjugacy classes of elements must be isomorphic to one of the following:

1. The cyclic group $C_3$ of order 3. 

2. The symmetric group $S_3$ of order 6.

3. The cyclic group $C_4$ of order 4.

4. The quaternion group $Q_8$ of order 8.

Proof:

Let $G$ be a finite group with exactly three conjugacy classes. Let $n_i$ denote the number of elements in the $i$-th conjugacy class ($1 \leq i \leq 3$) and $|G|$ denote the order of $G$. 

The class equation of a finite group $G$ is given by 
$$|G| = 1 + n_2 + n_3$$
where $1$ is the number of elements in the class of the identity element.

Since $G$ is a finite group, it has a finite number of elements, and hence $|G| = 1 + n_2 + n_3$ must be a positive integer.

By the Orthogonality Relations, each $n_i$ divides $|G|$. Thus, the possibilities for $(|G|, n_2, n_3)$ up to rearrangement are $(4, 1, 3)$, $(6, 1, 5)$, $(6, 2, 3)$, and $(8, 1, 7)$.

The first case corresponds to the cyclic group of order 4.

The second and third cases correspond to the symmetric group of order 6.

The fourth case corresponds to the quaternion group of order 8. 

The classification of groups of order $p^2$ (where $p$ is a prime) gives us that the only groups of order 4 are $C_4$ and $C_2	imes C_2$. But $C_2 	imes C_2$ has four conjugacy classes, so the group must be $C_4$. 

The classification of groups of order 6 gives us that the only groups of order 6 are $C_6$ and $S_3$. But $C_6$ has two conjugacy classes, so the group must be $S_3$.

The classification of groups of order 8 gives us that the only groups of order 8 are $C_8, C_4 	imes C_2, D_4$, and $Q_8$. But the first three groups do not have three conjugacy classes, so the group must be $Q_8$.

Therefore, the only finite groups with exactly three conjugacy classes are $C_3, S_3, C_4$, and $Q_8$.
\end{proof}

\begin{proof}Let us denote the three conjugacy classes by $C_1, C_2, C_3$. Note that the identity element forms a conjugacy class by itself, so we can assume $C_1 = \{e\}$.  Recall that in a finite group, the size of a conjugacy class divides the order of the group. Thus, $|C_2|$ and $|C_3|$ divide $|G|$.

Let $|C_2| = a$, and $|C_3| = b$. Then $|G| = a + b + 1$ and hence both $a$ and $b$ divide $a + b + 1$. This only happens if $a = b = 1$, or $a = 1, b = 2$, or $a = 2, b = 3$. 

The first case $a = b = 1$ gives us the trivial group $G = \{e\}$.

The second case $a = 1, b = 2$ gives us a group of order $4$. The only groups of order $4$ are $\mathbb{Z}_4$ and $\mathbb{Z}_2 	imes \mathbb{Z}_2$, but these have more than three conjugacy classes, so this case gives us no solutions.

The third case $a = 2, b = 3$ gives us a group of order $6$. The only groups of order $6$ are $\mathbb{Z}_6$ and $S_3$, but $\mathbb{Z}_6$ has $6$ conjugacy classes and $S_3$ has exactly three, namely $\{e\}$, $\{(12),(34)\}$ and $\{(123),(132)\}$. Hence, $S_3$ is a solution.

Therefore, the only finite groups with exactly three conjugacy classes are the trivial group and the symmetric group $S_3$.
\end{proof}

\begin{proof}First recall that the number of conjugacy classes of a group is equal to the number of irreducible representations of the group over the complex numbers (this is a result of the representation theory of finite groups). Furthermore, we know that the sum of the squares of the dimensions of the irreducible representations is equal to the order of the group. 

Assume that the dimensions of the irreducible representations of $G$ are $1, a, b$. The irreducible representation of dimension $1$ is the trivial representation. Then we must have $1 + a^2 + b^2 = |G|$. Moreover, $a, b$ must divide $|G|$ by the result that the degree of an irreducible character divides the group order. 

There are (up to isomorphism) three possibilities for $G$:

1. The cyclic group of order $3$. In this case, $a=b=1$ and the irreducible representations are all of dimension $1$.

2. The symmetric group $S_3$ of order $6$. In this case, $a=1, b=2$. The irreducible representations are of dimensions $1, 1, 2$.

3. The binary tetrahedral group, which is a subgroup of $SU(2)$, the group of $2	imes 2$ unitary matrices with determinant $1$. This group has $24$ elements, so $a=1, b=3$. The irreducible representations are of dimensions $1, 1, 3$.
\end{proof}
\end{document}
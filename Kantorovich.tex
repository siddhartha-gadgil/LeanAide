\documentclass[11pt]{article}

\usepackage{amsmath, amsthm, amssymb, fullpage}

\title{A Newton--Kantorovich Theorem with Surjective Derivative}
\author{}
\date{}

\newtheorem{theorem}{Theorem}

\begin{document}

\maketitle

\begin{abstract}
We state and prove a Newton--Kantorovich type theorem for functions 
$F : \mathbb{R}^n \to \mathbb{R}^m$ whose derivative at a point 
is surjective and admits a bounded right inverse.  The result gives
quantitative conditions ensuring the existence of a nearby zero,
uniqueness in an explicit ball, and convergence of a generalized Newton
iteration.
\end{abstract}

\section*{Theorem and Proof}

\begin{theorem}[Newton--Kantorovich with Surjective Derivative]
Let $F : \mathbb{R}^n \to \mathbb{R}^m$ be a $C^1$ function on an open set
containing $x_0$.  Assume:

\begin{enumerate}
\item[\textup{(1)}] (\textbf{Surjectivity and right inverse})  
$DF(x_0)$ is surjective, and there exists a linear map
\[
T : \mathbb{R}^m \to \mathbb{R}^n
\quad\text{such that}\quad 
DF(x_0) T = I_{\mathbb{R}^m}.
\]
Let $M := \|T\|$.

\item[\textup{(2)}] (\textbf{Lipschitz derivative})  
There exist constants $L \ge 0$ and $r>0$ such that
\[
\|DF(x) - DF(y)\| \le L \|x - y\|
\quad\text{for all } x,y \in B(x_0,r).
\]

\item[\textup{(3)}] (\textbf{Small defect})  
Let $\eta := \|T F(x_0)\|$ and define the Kantorovich parameter
\[
h := M L \eta.
\]
Assume
\[
h \le \frac12
\quad\text{and}\quad
\frac{1 - \sqrt{1 - 2h}}{M L} \le r.
\]
\end{enumerate}

Define
\[
\rho := \frac{1 - \sqrt{1 - 2h}}{M L}.
\]

Then:

\begin{enumerate}
\item[\textup{(a)}] There exists $x_* \in \mathbb{R}^n$ such that $F(x_*)=0$.
\item[\textup{(b)}] This point is unique in the ball $B(x_0,\rho)$.
\item[\textup{(c)}] The iteration
\[
x_{k+1} = x_k - T F(x_k), \qquad x_0 \text{ given},
\]
is well-defined for all $k$ and converges to $x_*$.
\item[\textup{(d)}] The solution lies within the explicit bound
\[
\|x_* - x_0\| \le \rho.
\]
\end{enumerate}
\end{theorem}

\begin{proof}
Define the Newton--Kantorovich map
\[
\Phi(x) := x - T F(x).
\]
A fixed point $x = \Phi(x)$ satisfies $T F(x) = 0$, and since $DF(x_0)T=I$,
this implies $F(x)=0$.  Thus zeros of $F$ in a set coincide with fixed
points of~$\Phi$.

\medskip
\noindent
\textbf{Step 1: Quadratic remainder estimate.}
For $x \in B(x_0,r)$, by Taylor's theorem and the Lipschitz bound on $DF$,
\[
F(x)
= F(x_0)
  + DF(x_0)(x-x_0)
  + R(x),
\]
where
\[
\|R(x)\|
\le
\frac{L}{2} \|x - x_0\|^2.
\tag{1}
\]

\medskip
\noindent
\textbf{Step 2: Cancellation of the linear part.}
For $x \in B(x_0,r)$,
\[
\Phi(x) - \Phi(x_0)
= (x-x_0) - T\bigl(F(x)-F(x_0)\bigr).
\]
Using the Taylor expansion,
\[
F(x) - F(x_0)
= DF(x_0)(x-x_0) + R(x),
\]
we obtain
\[
\Phi(x) - \Phi(x_0)
= (x-x_0)
  - T DF(x_0)(x-x_0)
  - T R(x).
\]
Since $T$ is a right inverse of $DF(x_0)$,
$DF(x_0) T = I_{\mathbb{R}^m}$, so the linear part cancels and
\[
\Phi(x) - \Phi(x_0)
= - T R(x).
\]
Using (1),
\[
\|\Phi(x)-\Phi(x_0)\|
\le M \frac{L}{2} \|x-x_0\|^2.
\tag{2}
\]

\medskip
\noindent
\textbf{Step 3: Mapping a ball into itself.}
Let $\rho > 0$.  For $x \in B(x_0,\rho)$,
\[
\|\Phi(x)-x_0\|
\le
\|\Phi(x)-\Phi(x_0)\|
+ \|\Phi(x_0)-x_0\|.
\]
Since $\Phi(x_0)-x_0 = -T F(x_0)$, we have $\|\Phi(x_0)-x_0\|=\eta$.
Using (2),
\[
\|\Phi(x)-x_0\|
\le
\eta + \frac{M L}{2}\rho^2.
\tag{3}
\]

Thus $\Phi$ maps $B(x_0,\rho)$ into itself provided
\[
\eta + \frac{M L}{2}\rho^2 \le \rho.
\tag{4}
\]

This is a quadratic inequality in $\rho$:
\[
\frac{M L}{2}\rho^2 - \rho + \eta \le 0.
\]

Solving the corresponding quadratic equation, the discriminant is
$\Delta = 1 - 2 M L \eta = 1 - 2h$.  Under the hypothesis $h \le 1/2$,
$\Delta \ge 0$, and the smaller root is
\[
\rho = \frac{1 - \sqrt{1 - 2h}}{M L}.
\tag{5}
\]
For this choice, (4) holds with equality.  The assumption $\rho \le r$
ensures $B(x_0,\rho)$ lies inside the region where the Lipschitz bound holds.

\medskip
\noindent
\textbf{Step 4: Contraction property.}
Differentiating $\Phi$ gives
\[
D\Phi(z)
= I - T DF(z).
\]
Thus
\[
\|D\Phi(z)\|
= \|T(DF(x_0)-DF(z))\|
\le M L \|z - x_0\|
\le M L \rho.
\]
For the value of $\rho$ in (5),
\[
M L \rho
= 1 - \sqrt{1 - 2h}
< 1.
\]
Hence $\Phi$ is a contraction on the closed ball $\overline{B}(x_0,\rho)$.

\medskip
\noindent
\textbf{Step 5: Existence and uniqueness of the zero.}
Since $\overline{B}(x_0,\rho)$ is complete and $\Phi$ is a contraction
mapping it to itself, the Banach fixed point theorem yields a unique
fixed point $x_*$ in the ball.  
This fixed point satisfies $F(x_*) = 0$.  
Uniqueness in the ball follows immediately.

\medskip
\noindent
\textbf{Step 6: Convergence of the iteration.}
The iteration $x_{k+1} = \Phi(x_k)$ stays inside $B(x_0,\rho)$
and converges to the unique fixed point $x_*$. 
Finally,
\[
\|x_* - x_0\| \le \rho.
\]

\end{proof}

\end{document}
